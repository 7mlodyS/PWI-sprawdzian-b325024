\documentclass[a4paper, 11pt]{article}
\usepackage[T1]{fontenc}
\usepackage[utf8]{inputenc}
\usepackage[polish]{babel}
\usepackage{amsmath}
\usepackage{amssymb}
\usepackage[nofoot,hdivide={2cm,*,1cm},vdivide={1cm,*,1cm}]{geometry}
\pagestyle{empty}
\author{Stefan Bogdan}
\title{Sprawozdanie z kolokwium}
\date{\today}


\begin{document}
\maketitle
\section*{Zadanie 1}
\begin{itemize}
    \item $ \rho \frac{D \textbf{u}}{Dt} = \rho \big( \frac{\partial \textbf{u}}{\partial t} + \textbf{u} \cdot \nabla
            \textbf{u} \big) = - \nabla \overline{p} + \nabla \cdot \big \{ \mu \big( \nabla \textbf{u} + \big( \nabla 
            \textbf{u} \big) ^\text{T} - \frac{2}{3} \big( \nabla \cdot \textbf{u} \big) \textbf{I} \big) \big \} + \rho
            \textbf{g} $
            
    \item $ \tilde{f}(\xi) = \int^\infty_{-\infty} f(x) e^{-2\pi ix \xi} dx $
    
    \item $ \mathbb{P} \big( \hat{X_n} - z_{1 - \frac{\alpha}{2}} \frac{\sigma}{\sqrt{n}} \leqslant \mathbb{E}X 
            \leqslant \hat{X_n} + z_{1 - \frac{\alpha}{2}} \frac{\sigma}{\sqrt{n}} \big) \approx 1 - \alpha $
            
    \item $ \begin{bmatrix} 1 & 2 \\ 3 & 4 \end{bmatrix} \otimes \begin{bmatrix} 0 & 5 \\ 6 & 7 \end{bmatrix} = 
            \begin{bmatrix} 
            1 \begin{bmatrix} 0 & 5 \\ 6 & 7 \end{bmatrix} & 2 \begin{bmatrix} 0 & 5 \\ 6 & 7 \end{bmatrix} \\
            3 \begin{bmatrix} 0 & 5 \\ 6 & 7 \end{bmatrix} & 4 \begin{bmatrix} 0 & 5 \\ 6 & 7 \end{bmatrix}
            \end{bmatrix} =
            \begin{bmatrix} 0 & 5 & 0 & 10 \\ 6 & 7 & 12 & 14 \\ 0 & 15 & 0 & 20 \\ 18 & 21 & 24 & 28 \end{bmatrix}
          $
\end{itemize}

\section*{Zadanie 2}

\subsection*{2.1. Generowanie klucza}
\begin{verbatim}
ssh-keygen -t rsa -C "stefan7bogdan@gmail.com"
\end{verbatim}
\begin{itemize}
    \item \verb+ssh-keygen+ to komenda generująca klucz
    \item flaga \verb+t+ określa typ klucza - tutaj \verb+rsa+
    \item flaga \verb+C+ dodaje komentarz, w cudzysłowach podaję nazwę użytkownika, z kórym ma być powiązany ten klucz
\end{itemize}
Generuję takie dwa klucze - jeden do repozytorium i jeden do serwera.

\subsection*{2.2. Przenoszenie klucza na zdalny serwer}
\begin{verbatim}
ssh-copy-id -i ~/.ssh/id_pwi_rsa b325024@pwi.ii.uni.wroc.pl
\end{verbatim}
\begin{itemize}
    \item \verb+ssh-copy-id+ to komenda kopiująca klucz na serwer
    \item flaga \verb+i+ określa plik z którego klucz ma być przekopiowany, podaję do niego ścieżkę
    \item na koniec jest \verb+user@host+
\end{itemize}
Wpisuję hasło i klucze znajdują się już na serwrze.\\

\subsection*{2.3. Tworzenie repozytorium na GitHubie}
Wchodzę na GitHuba i wybieram odpowiednio:
\begin{verbatim}
Ikonka mojego profilu -> Your repositories -> New -> wpisuję nazwę -> Create 
\end{verbatim}
i w taki sposób mam stworzone nowe puste repozytorium.

\noindent Wchodzę do niego i wybieram:
\begin{verbatim}
Settings -> Deploy keys -> wpisuję tytuł i kopiuję klucz publiczny -> Add key
\end{verbatim}
i w taki sposób mam dodany klucz na moje repozytorium.\\

\subsection*{2.4. Tworzenie pliku konfiguracyjnego}
\begin{verbatim}
cd .ssh
touch config
\end{verbatim}
Wygląd pliku \verb+config+:
\begin{verbatim}
Host pwi-sprawdzian
        Hostname pwi.ii.uni.wroc.pl
        User b325024
        Identityfile ~/.ssh/id_pwi_rsa
\end{verbatim}
\begin{itemize}
    \item pierwsza linijka wskazuje na to czego będzie używać ssh kiedy wywoła się go komendę \verb+ssh pwi-sprawdzian+
    \item \verb+Hostname+ to nazwa hosta, z którym będziemy się łączyć
    \item \verb+User+ to nazwa użytkownika, przez którego będziemy się łączyć
    \item \verb+Identityfile+ to ścieżka to pliku, w którym jest nasz klucz prywatny
\end{itemize}

Teraz mogę połączyć się z serwerem wydając tylko polecenie \verb+ssh pwi-sprawdzian+

\subsection*{2.5. Przekierowanie klucza lokalnego}
Żeby przekierować klucz lokalny trzeba dodać do pliku \verb+config+ linijkę \verb+ForwardAgent yes+ co będzie oznaczało, że będziemy uzywać agetna ssh.

\section*{Zadanie 3}

\subsection*{3.1. Klonowanie repozytorium}
Najpierw muszę odpalić agenta ssh, żeby mógł \emph{przeforwardować} klucz z mojego komputera gdzieś dalej. Jeśli tego nie zrobię, mogę \emph{brzydko} wygenerować
kolejny klucz na zdalnym komputerze. Jest to \emph{brzydkie} rozwiązanie, ponieważ dostęp do tego klucza na serwerze może mieć ktoś inny oprócz mnie
i może się podemnie podszywać. Dodaję więc mój klucz do agenta ssh komendą \verb+ssh-add ~/.ssh/id_github_rsa+, podając ścieżkę do mojego klucza.\\
Loguję się na serwer za pomocą komendy\\
\verb+ssh pwi-sprawdzian+\\ 
i klonuje moje repozytorium komendą\\
\verb+git clone git@github.com:7mlodyS/PWI-sprawdzian-b325024.git+

\subsection*{3.2 Pobieranie i skomitowanie zmian}
Będąc w koatalogu z moim rezpozytorium pobieram plik poleceniem \\
\verb+wget http://www.ii.uni.wroc.pl/~lisu/zadanie.tar.gz+ \\
Wypakowuję go komendą 
\verb+tar -xf zadanie.tar.gz+
\begin{itemize}
    \item flaga \verb+x+ od \emph{extract} rozpakowuje pliki w archiwum
    \item flaga \verb+f+ od \emph{file} mówi, że ma rozpakować później podany plik, czyli \verb+zadanie.tar.gz+
\end{itemize}
Żeby skomitować zmiany wykonuję komendy:
\begin{verbatim}
git add .
git commit -m "pierwszy"
\end{verbatim}

\subsection*{3.3.1 Wyszukanie zadania}
Wyliczam funkcję skórtu \verb+MD5+ ze stringa \verb+b325024+ komendą \\ 
\verb+echo -n b325024 | md5sum+
\begin{itemize}
    \item flaga \verb+n+ nie dodaje znaku końca lini po wypisanym tekscie
\end{itemize}

\noindent Dostałem: \verb+50da5909fe4f874dfd1a3876c7bf7fd7+
Wyszukuję taki folder komendą:\\
\verb+find -type d -name "50da5909fe4f874dfd1a3876c7bf7fd7"+
\begin{itemize}
    \item flaga \verb+-type+ określa typ, a \verb+d+ to \emph{directory}, czyli folder
    \item flaga \verb+-name+ określa nazwę którą poźniej podaję pomiędzy \verb+" "+
\end{itemize}

\subsection*{3.3.2 Wykonanie zadania}
Szukam użytkowników z Polski komendą
\begin{verbatim}
grep -xc "[a-zA-Z@.:0-9]* | Country = POLAND .*" users.db
\end{verbatim}
\begin{itemize}
    \item flaga \verb+x+ dopasowuje wyrażenie dokładnie do całej linii
    \item flaga \verb+c+ liczy ilość wystąpień wyrażenia
\end{itemize}
Liczę wszystkich użytkowników komendą
\begin{verbatim}
grep -xc ".*" users.db
\end{verbatim}
Wyliczam ich stosunek w pythonie. Wynik to \verb+0.013893032175580652+
\\ 
\\
Wycinam najpierw część znaków przed hasłem komendą i zapisuję je do pliku pomocniczego
\begin{verbatim}
sed 's/[^:]*://' users.db > pomocniczy.txt
\end{verbatim}
Teraz wycinam znaki po haśle komendą i zapisuję do pliku \verb+passwords.txt+
\begin{verbatim}
sed 's/ |.*//' pomocniczy.txt > passwords.txt
\end{verbatim}

\section*{Zadanie 4}
Kopiuję plik \verb+sprawozdanie.tex+ na zdalne repozytorium komendą
\begin{verbatim}
scp sprawozdanie.pdf b325024@pwi.ii.uni.wroc.pl:PWI-sprawdzian-b325024
\end{verbatim}



\section*{bibliografia}
\begin{verbatim}
https://oeis.org/wiki/List_of_LaTeX_mathematical_symbols
\end{verbatim}

\end{document}
